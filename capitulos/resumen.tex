%chapter introduce un nuevo capítulo
\chapter{Resumen}

En un sistema BIPV de ventana compacto dotado de un vidrio externo con un conjunto de lentes de Fresnel que concentran la radiación solar directa en líneas focales horizontales, es necesario un control preciso de un plano móvil situado en el interior del marco de la ventana. Este plano permite hacer coincidir células solares dispuestas transversalmente al panel con dichas líneas focales a lo largo del día, siguiendo el movimiento aparente del Sol. Asimismo, el sistema permite regular la cantidad y el tipo de luz que se transmite al interior del edificio.

Este trabajo presenta la arquitectura de un programa de control en lazo abierto que, introduciendo únicamente la localización geográfica del sistema y la orientación del panel (azimut del panel e inclinación), calcula las posiciones objetivo y gobierna el movimiento del plano móvil para realizar un seguimiento solar efectivo.


\paragraph{Palabras clave:} BIPV, concentración solar, lentes de Fresnel, seguimiento solar, ventanas fotovoltaicas, control en lazo abierto.

\chapter{Abstract}



\paragraph{Keywords:} keyword1, keyword2, keyword3.